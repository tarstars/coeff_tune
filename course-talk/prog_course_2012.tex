\documentclass{beamer}
\usepackage[utf8x]{inputenc}
\usepackage[english,russian]{babel}
\usepackage{beamerthemesplit}

\usepackage{color}
\definecolor{gray}{rgb}{0.4,0.4,0.4}
\definecolor{bggray}{rgb}{0.9,0.9,0.9}
%	Параметры подсветки исходного кода на C++
\usepackage{listings}
\lstset{breakatwhitespace,
	language=C++,
	columns=fullflexible,
	keepspaces,
	breaklines,
	tabsize=3,
	numbers=left, % нумерация строк
	numberstyle=\small\color{gray},
	numbersep=5pt,
	backgroundcolor=\color{bggray},
	showstringspaces=false,
	extendedchars=true,
	keywordstyle=\color{blue}\ttfamily,
	stringstyle=\color{red}\ttfamily,
	commentstyle=\color{green}\ttfamily,
	morecomment=[l][\color{magenta}]{\#}
}


\begin{document}
\title[Численные методы акустооптики на C++]{Отчёт о ведении спецкурса \\"Численные методы в приложении к задачам акустооптики и акустоэлектроники на языке программирования С++" в 2012 году}
\author{Арсений Трушин\texttrademark \& \(Co\)} 
\date{\today}

%% %	Пример листинга на C++, отступы, включая отступы от начала строки сохраняются (!)
%% \defverbatim[colored]\lst{
%%   \begin{lstlisting}[tabsize=2,basicstyle=\ttfamily]
%%     const char *processing() const{
%%       somestring = "-1";
%%     }
%% \end{lstlisting}}

%% %	Вставка листинга на C++
%% \frametitle{Пример кода на C++}
%% \lst

%	Слайд-заголовок (#1)
\frame{\titlepage}
%	Слайд с содержанием презентации (#2)
\frame{\frametitle{Содержание}\tableofcontents} 

\section{Обзор. Цели и методы спецкурса.}
%\subsection{Проблема}
\frame{
  \frametitle{Преподавание программирования на Физфаке}
  \begin{block}{Первые курсы}
    \begin{itemize}
    \item На первых курсах есть C++
    \item Рассказывают о Win32API
    \item Не рассказывают о STL
    \end{itemize}
  \end{block}
  \begin{block}{В рамках спецкурса}
    \begin{itemize}
    \item Решение практических задач
    \item ``Точки входа'' в полезные технологии
    \item Решения в ``две строчки''
    \end{itemize}
  \end{block}
}
%
\frame{
  \frametitle{Особенности спецкурса}
  \begin{block}{Организационные}
    \begin{itemize}
    \item Маленькая группа - адаптивность
    \item Нацеленность на актуальные задачи
    \end{itemize}
  \end{block}
%
  \begin{block}{Технологии}
    \begin{itemize}
      \item g++/mingw
      \item emacs
      \item git
    \end{itemize}
  \end{block}
}
%
\subsection{}\frame{}
%
\subsection{PovRay}\frame{}
%
\section{Постановка задачи}
\frame{
  \frametitle{...}
  When I find my code in tons of trouble,\\
  Friends and colleagues come to me,\\
  Speaking words of wisdom:\\
  "Write in C."
}
\subsection{Уравнения Кристоффеля}
\subsection{Экстракция экспериментальных данных}
\subsection{???}
\subsection{PROFIT!}

%Мобилограмма
% Tools:
% - C++ (С)
% - Qt (С)
% - mingw (С)
% - emacs (С)
% - git (С)
% - PovRay (П)
% Подбор констант:
% - кристоффель (Ю)
% - экстракция скоростей из файла (Ю)
% - градиентный спуск (П)
% - рой (Н)
% - отжиг (С)
% - дифракция (Д)

\section{Подбор констант}
	\subsection{Метод отжига}\frame{}
	\subsection{Градиентный спуск}\frame{}
	\subsection{``Рой''}\frame{}
	
%\frame{
%\subsection{сПиСоК}
%\frame{\frametitle{unnumbered lists}
%\begin{itemize}
%\item Как выстрелить себе в ногу при помощи C++ и \LaTeX  
%\item *Курс для продвинутых
%\item Быдлокод
%\item BDD, TDD, Agile, XP и другие страшные слова
%\end{itemize} 
%}
%
%\frame{\frametitle{lists with pause}
%\begin{itemize}
%\item Introduction to  \LaTeX \pause 
%\item Course 2 \pause 
%\item Termpapers and presentations with \LaTeX \pause 
%\item Beamer class
%\end{itemize} 
%}
%
%\subsection{Lists II}
%\frame{\frametitle{numbered lists}
%\begin{enumerate}
%\item Introduction to  \LaTeX  
%\item Course 2 
%\item Termpapers and presentations with \LaTeX 
%\item Beamer class
%\end{enumerate}
%}
%\frame{\frametitle{numbered lists with pause}
%\begin{enumerate}
%\item Introduction to  \LaTeX \pause 
%\item Course 2 \pause 
%\item Termpapers and presentations with \LaTeX \pause 
%\item Beamer class
%\end{enumerate}
%}

%\section{Section no.3} 
%\subsection{Tables}
%\frame{\frametitle{Tables}
%\begin{tabular}{|c|c|c|}
%\hline
%\textbf{Date} & \textbf{Instructor} & \textbf{Title} \\
%\hline
%WS 04/05 & Sascha Frank & First steps with  \LaTeX  \\
%\hline
%SS 05 & Sascha Frank & \LaTeX \ Course serial \\
%\hline
%\end{tabular}}
%
%
%\frame{\frametitle{Tables with pause}
%\begin{tabular}{c c c}
%A & B & C \\ 
%\pause 
%1 & 2 & 3 \\  
%\pause 
%A & B & C \\ 
%\end{tabular} }
%
%
%\section{Section no. 4}
%\subsection{blocs}
%\frame{\frametitle{blocs}
%
%\begin{block}{title of the bloc}
%bloc text
%\end{block}
%
%\begin{exampleblock}{title of the bloc}
%bloc text
%\end{exampleblock}
%
%
%\begin{alertblock}{title of the bloc}
%bloc text
%\end{alertblock}
%}
\end{document}

